\chapter{troisième chapitre}

\section{Informatique}

Vous pouvez inclure du code de la manière suivante :

\begin{minted}[frame=single,framesep=2mm,baselinestretch=1,fontsize=\footnotesize,linenos]{cpp}
{
  #include <iostream>

  int main() {
      std::cout << "Hello World!";
      return 0;
  }
}
\end{minted}

\section{Chimie}

Vous pouvez inclure des formules chimiques de la manière suivante :

\begin{figure}[H]
  \centering
  \chemfig{-[:30](=[:90]O)-[:-30]OH}
  \caption{La formule de l'Éthanol.}
  \label{fig:ethanol}
\end{figure}

\noindent La formule de l'Éthanol est représentée en Figure \ref{fig:ethanol}.

\section{Ingénierie}

Vous pouvez inclure des schémas électriques de la manière suivante :

\begin{figure}[H]
  \centering
  \begin{circuitikz}[american voltages]
    \draw
      (0,0) to [short, *-] (6,0)
      to [V, l_=$\mathrm{j}{\omega}_m \underline{\psi}^s_R$] (6,2)
      to [R, l_=$R_R$] (6,4)
      to [short, i_=$\underline{i}^s_R$] (5,4)

      (0,0) to [open, v^>=$\underline{u}^s_s$] (0,4)
      to [short, *- ,i=$\underline{i}^s_s$] (1,4)
      to [R, l=$R_s$] (3,4)
      to [L, l=$L_{\sigma}$] (5,4)
      to [short, i_=$\underline{i}^s_M$] (5,3)
      to [L, l_=$L_M$] (5,0);
  \end{circuitikz}
  \caption{Exemple d'un schéma électriques.}
  \label{fig:elec}
\end{figure}

\noindent La Figure \ref{fig:elec} montre un exemple de schéma électriques.

\section{Physique}

Vous pouvez inclure des diagrammes de Feynman de la manière suivante :

\begin{figure}[H]
  \centering
  \feynmandiagram [small, horizontal=a to t1] {
    a [particle=\(\pi^{0}\)] -- [scalar] t1 -- t2 -- t3 -- t1,
    t2 -- [photon] p1 [particle=\(\gamma\)],
    t3 -- [photon] p2 [particle=\(\gamma\)],
  };
  \caption{Exemple d'un diagramme de Feynman.}
  \label{fig:feynman}
\end{figure}

\noindent La Figure \ref{fig:feynman} montre l'exemple d'un diagramme de Feynman.
