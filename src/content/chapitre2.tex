\chapter{second chapitre}

\section{exemples avancés}

\subsection{équations}

\begin{equation}
  \label{eq:eq_1}
  E=mc^2
\end{equation}

Il est possible de référencer l'Équation \ref{eq:eq_1} dans le texte.

\subsection{théorèmes et preuves}

\begin{theorem}
  \label{theo:theo_1}
  Let $f$ be a function whose derivative exists in every point, then $f$ is
  a continuous function.
\end{theorem}

\begin{theorem}[Pythagorean theorem]
  \label{theo:theo_2}
  This is a theorema about right triangles and can be summarised in the next
  equation
  \[ x^2 + y^2 = z^2 \]
\end{theorem}

\begin{corollary}
  \label{coro:coro_1}
  There's no right rectangle whose sides measure 3cm, 4cm, and 6cm.
\end{corollary}

\begin{lemma}
  \label{lem:lem_1}
  Given two line segments whose lengths are $a$ and $b$ respectively there
  is a real number $r$ such that $b=ra$.
\end{lemma}

\begin{definition}[Fibration]
  \label{def:def_1}
  A fibration is a mapping between two topological spaces that has the homotopy lifting property for every space $X$.
\end{definition}

\begin{proof}
  \label{proof:proof_1}
  To prove it by contradiction try and assume that the statement is false,
  proceed from there and at some point you will arrive to a contradiction.
\end{proof}

Vous pouvez ensuite référencer les Théorèmes \ref{theo:theo_1} et \ref{theo:theo_2}, ainsi que le Corollaire \ref{coro:coro_1} et le Lemme \ref{lem:lem_1} dans votre texte. De plus, la Définition \ref{def:def_1} et la Démonstration \ref{proof:proof_1} peuvent également être référencées dans le texte.

\subsection{algorithmes}

\begin{algorithm}[H]
  \KwData{input}
  \KwResult{output}
  initialization\;
   \While{predicat}{
    instructions\;
    \eIf{condition}{
     instructions1\;
     instructions2\;
     }{
     instructions3\;
    }
   }
  \caption{Titre de l'algorithme.}
  \label{algo:algo_1}
\end{algorithm}

L'algorithme \ref{algo:algo_1} peut ensuite être référencé dans le texte.
