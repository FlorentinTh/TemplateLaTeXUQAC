%%%%%%%%%%%%%%%%%%%%%%%%%%%%%%%%%%%%%%%%%%%%%%%%%%%%%%%%%%%%%%%%%%%%%%%%%%%%%
%%
%% Classes disponibles :
%%	- times : utilisation de la police Times par défaut.
%%	- these, sujthese, projthese, memoire, essai, rapport : pour
%%		générer la page de garde en fonction du type document.
%%	- french, english : selon la langue dans laquelle doit
%%		être rédigé le document.
%%  - ieee, apa : style des références bibliographiques
%%  - twoside : alternance des marges pour impression recto/verso
%%    vous devez décommenter la ligne 16 si vous utilisez cette classe
%%  - rev : permet d'activer l'identification des URLs pour la
%%    révision bibliographique. Vous ne devez pas faire votre dépôt final
%%    sans avoir préalablement retiré l'utilisation de cette classe.
%%
%%%%%%%%%%%%%%%%%%%%%%%%%%%%%%%%%%%%%%%%%%%%%%%%%%%%%%%%%%%%%%%%%%%%%%%%%%%%%%
\documentclass[12pt,times,these,french,apa]{uqac}

%% Décommenter la ligne suivante avec l'utilisation de la classe "twoside"
% \raggedbottom

%% Préciser l'emplacement du fichier contenant les acronymes.
\acrolistpath{assets/acro}

%% Toutes les images utilisées doivent se trouver dans le répertoire "assets/figure".
\graphicspath{{assets/figures/}}

\begin{document}

% Titre du document
\title{Titre de la thèse du mémoire ou de l'essai}

% Auteur du document
\author{Prénom Nom}

%% Préciser le nom du programme
\programme{Informatique}

%% S’il y a lieu,
%% indiquer le profil
%% ou la concentration du programme
%\concentration{profil recherche}

%% Par défaut l'année courante est utilisée.
%% Pour spécifier une autre année :
%\degreeyear{2018}

\maketitle

%%%%%%%%%%%%%%%%%%%%%
%%
%% Page préliminaires
%%
%%%%%%%%%%%%%%%%%%%%%
\opening

\include{content/abstract}

%% Si vous écrivez un document en anglais,
%% il sera nécessaire de fournir un résumé en Français :

%\include{content/resume}

\tableofcontents
\cleardoublepage
\listoftables
\listoffigures
\listofacro

\include{content/dedicace}
\include{content/remerciements}
\include{content/avant_propos}

%%%%%%%%%%%%%%%%%%%%%
%%
%% Document principal
%%
%%%%%%%%%%%%%%%%%%%%%
\maincontent

\include{content/introduction}
\chapter{premier chapitre}
	
\section{exemples} 

\subsection{citations} 

Ceci est un premier exemple de citation \cite{Ful83}. Ceci est un deuxième exemple de citation \citep{GMP81}. Ceci est un troisième exemple de citation \citet{PP98}.

\subsection{acronymes} 

Exemple de définition d'un acronyme de trois lettres : \ac{TLA}. Puis utilisation de cet acronyme en version courte \acs{TLA}.

\subsection{equations}   

\begin{equation}
E=mc^2	
\end{equation}

\subsection{figures}   

Lorem ipsum dolor sit amet, consectetur adipisicing elit, sed do eiusmod
tempor incididunt ut labore et dolore magna aliqua. Ut enim ad minim veniam,
quis nostrud exercitation ullamco laboris nisi ut aliquip ex ea commodo
consequat.

Lorem ipsum dolor sit amet, consectetur adipisicing elit, sed do eiusmod
tempor incididunt ut labore et dolore magna aliqua. Ut enim ad minim veniam,
quis nostrud exercitation ullamco laboris nisi ut aliquip ex ea commodo
consequat.

\begin{figure}[H]
 \centering
 \includegraphics[width=7cm]{lenna.png}
 \caption{Titre de la figure}
 \label{fig:figure1}
\end{figure}

Lorem ipsum dolor sit amet, consectetur adipisicing elit, sed do eiusmod
tempor incididunt ut labore et dolore magna aliqua. Ut enim ad minim veniam,
quis nostrud exercitation ullamco laboris nisi ut aliquip ex ea commodo
consequat.

Lorem ipsum dolor sit amet, consectetur adipisicing elit, sed do eiusmod
tempor incididunt ut labore et dolore magna aliqua. Ut enim ad minim veniam,
quis nostrud exercitation ullamco laboris nisi ut aliquip ex ea commodo
consequat.

\subsection{tableaux}   

Lorem ipsum dolor sit amet, consectetur adipisicing elit, sed do eiusmod
tempor incididunt ut labore et dolore magna aliqua. Ut enim ad minim veniam,
quis nostrud exercitation ullamco laboris nisi ut aliquip ex ea commodo
consequat.

\begin{table}[H]
  \begin{center}
    \caption{Titre du tableau}
    \label{tab:table1}
    \begin{tabular}{l|c|r}
      \textbf{Value 1} & \textbf{Value 2} & \textbf{Value 3}\\
      $\alpha$ & $\beta$ & $\gamma$ \\
      \hline
      1 & 1110.1 & a\\
      2 & 10.1 & b\\
      3 & 23.113231 & c\\
    \end{tabular}
  \end{center}
\end{table}

Lorem ipsum dolor sit amet, consectetur adipisicing elit, sed do eiusmod
tempor incididunt ut labore et dolore magna aliqua. Ut enim ad minim veniam,
quis nostrud exercitation ullamco laboris nisi ut aliquip ex ea commodo
consequat.
\chapter{second chapitre}
	
\section{sous-titre} 

Lorem ipsum dolor sit amet, consectetur adipisicing elit, sed do eiusmod
tempor incididunt ut labore et dolore magna aliqua. Ut enim ad minim veniam,
quis nostrud exercitation ullamco laboris nisi ut aliquip ex ea commodo
consequat. Duis aute irure dolor in reprehenderit in voluptate velit esse
cillum dolore eu fugiat nulla pariatur. Excepteur sint occaecat cupidatat non
proident, sunt in culpa qui officia deserunt mollit anim id est laborum.

\subsection{sous-titre} 

Lorem ipsum dolor sit amet, consectetur adipisicing elit, sed do eiusmod
tempor incididunt ut labore et dolore magna aliqua. Ut enim ad minim veniam,
quis nostrud exercitation ullamco laboris nisi ut aliquip ex ea commodo
consequat. Duis aute irure dolor in reprehenderit in voluptate velit esse
cillum dolore eu fugiat nulla pariatur. Excepteur sint occaecat cupidatat non
proident, sunt in culpa qui officia deserunt mollit anim id est laborum.

\subsection{sous-titre}   

Lorem ipsum dolor sit amet, consectetur adipisicing elit, sed do eiusmod
tempor incididunt ut labore et dolore magna aliqua. Ut enim ad minim veniam,
quis nostrud exercitation ullamco laboris nisi ut aliquip ex ea commodo
consequat. Duis aute irure dolor in reprehenderit in voluptate velit esse
cillum dolore eu fugiat nulla pariatur. Excepteur sint occaecat cupidatat non
proident, sunt in culpa qui officia deserunt mollit anim id est laborum.

\begin{table}[H]
  \begin{center}
    \caption{Titre du tableau}
    \label{tab:table2}
    \begin{tabular}{l|c|r}
      \textbf{Value 1} & \textbf{Value 2} & \textbf{Value 3}\\
      $\alpha$ & $\beta$ & $\gamma$ \\
      \hline
      1 & 1110.1 & a\\
      2 & 10.1 & b\\
      3 & 23.113231 & c\\
    \end{tabular}
  \end{center}
\end{table}

Lorem ipsum dolor sit amet, consectetur adipisicing elit, sed do eiusmod
tempor incididunt ut labore et dolore magna aliqua. Ut enim ad minim veniam,
quis nostrud exercitation ullamco laboris nisi ut aliquip ex ea commodo
consequat. Duis aute irure dolor in reprehenderit in voluptate velit esse
cillum dolore eu fugiat nulla pariatur. Excepteur sint occaecat cupidatat non
proident, sunt in culpa qui officia deserunt mollit anim id est laborum.

\section{sous-titre}   

Lorem ipsum dolor sit amet, consectetur adipisicing elit, sed do eiusmod
tempor incididunt ut labore et dolore magna aliqua. Ut enim ad minim veniam,
quis nostrud exercitation ullamco laboris nisi ut aliquip ex ea commodo
consequat. Duis aute irure dolor in reprehenderit in voluptate velit esse
cillum dolore eu fugiat nulla pariatur. Excepteur sint occaecat cupidatat non
proident, sunt in culpa qui officia deserunt mollit anim id est laborum.
             
Lorem ipsum dolor sit amet, consectetur adipisicing elit, sed do eiusmod
tempor incididunt ut labore et dolore magna aliqua. Ut enim ad minim veniam,
quis nostrud exercitation ullamco laboris nisi ut aliquip ex ea commodo
consequat. Duis aute irure dolor in reprehenderit in voluptate velit esse
cillum dolore eu fugiat nulla pariatur. Excepteur sint occaecat cupidatat non
proident, sunt in culpa qui officia deserunt mollit anim id est laborum.
\chapter{troisième chapitre}

\section{Informatique}

Vous pouvez inclure du code de la manière suivante :

\begin{minted}[frame=single,framesep=2mm,baselinestretch=1,fontsize=\footnotesize,linenos]{cpp}
{
  #include <iostream>

  int main() {
      std::cout << "Hello World!";
      return 0;
  }
}
\end{minted}

\section{Chimie}

Vous pouvez inclure des formules chimiques de la manière suivante :

\begin{figure}[H]
  \centering
  \chemfig{-[:30](=[:90]O)-[:-30]OH}
  \caption{La formule de l'Éthanol.}
  \label{fig:ethanol}
\end{figure}

\noindent La formule de l'Éthanol est représentée en Figure \ref{fig:ethanol}.

\section{Ingénierie}

Vous pouvez inclure des schémas électriques de la manière suivante :

\begin{figure}[H]
  \centering
  \begin{circuitikz}[american voltages]
    \draw
      (0,0) to [short, *-] (6,0)
      to [V, l_=$\mathrm{j}{\omega}_m \underline{\psi}^s_R$] (6,2)
      to [R, l_=$R_R$] (6,4)
      to [short, i_=$\underline{i}^s_R$] (5,4)

      (0,0) to [open, v^>=$\underline{u}^s_s$] (0,4)
      to [short, *- ,i=$\underline{i}^s_s$] (1,4)
      to [R, l=$R_s$] (3,4)
      to [L, l=$L_{\sigma}$] (5,4)
      to [short, i_=$\underline{i}^s_M$] (5,3)
      to [L, l_=$L_M$] (5,0);
  \end{circuitikz}
  \caption{Exemple d'un schéma électriques.}
  \label{fig:elec}
\end{figure}

\noindent La Figure \ref{fig:elec} montre un exemple de schéma électriques.

\section{Physique}

Vous pouvez inclure des diagrammes de Feynman de la manière suivante :

\begin{figure}[H]
  \centering
  \feynmandiagram [small, horizontal=a to t1] {
    a [particle=\(\pi^{0}\)] -- [scalar] t1 -- t2 -- t3 -- t1,
    t2 -- [photon] p1 [particle=\(\gamma\)],
    t3 -- [photon] p2 [particle=\(\gamma\)],
  };
  \caption{Exemple d'un diagramme de Feynman.}
  \label{fig:feynman}
\end{figure}

\noindent La Figure \ref{fig:feynman} montre l'exemple d'un diagramme de Feynman.

\include{content/conclusion}

%%%%%%%%$%%%%
%%
%% Références
%%
%%%%%%%%%%%%%

%% Le fichier bibtex doit se trouver dans le répertoire "assets/references".
\bibliography{assets/references}

%%%%%%%%%%%
%%
%% Annexes
%%
%%%%%%%%%%%
\appendix

\include{content/annexe_a}

\end{document}
